\documentclass{article}

\usepackage{Sweave}
\begin{document}
\Sconcordance{concordance:perron_frob.tex:perron_frob.Rnw:%
1 2 1 1 0 4 1 1 2 1 0 2 1 8 0 2 1 5 0 1 1 8 0 2 1 5 0 2 1 5 0 11 1 8 0 %
1 1 8 0 1 1 8 0 1 1 8 0 1 1 8 0 1 1 8 0 1 1 8 0 1 1 8 0 1 1 8 0 1 1 9 0 %
1 2 2 1 1 2 1 0 1 1 9 0 1 2 4 1}


To grasp the Perron-Frobenius theorem:

\begin{Schunk}
\begin{Sinput}
> library(expm)
> A <- matrix(c(1, -3, -6, 3, 7, 6, -3,-3,-2), nrow=3)
> A
\end{Sinput}
\begin{Soutput}
     [,1] [,2] [,3]
[1,]    1    3   -3
[2,]   -3    7   -3
[3,]   -6    6   -2
\end{Soutput}
\begin{Sinput}
> r <- eigen(A)
> r$values
\end{Sinput}
\begin{Soutput}
[1]  4  4 -2
\end{Soutput}
\begin{Sinput}
> r$vectors
\end{Sinput}
\begin{Soutput}
           [,1]      [,2]       [,3]
[1,] -0.8164950 0.3487429 -0.4082483
[2,] -0.4068588 0.8137335 -0.4082483
[3,]  0.4096362 0.4649906 -0.8164966
\end{Soutput}
\begin{Sinput}
> maxlambda <- max(r$values)
> maxlambda
\end{Sinput}
\begin{Soutput}
[1] 4
\end{Soutput}
\begin{Sinput}
> spect <- max(abs(r$values))
> spect
\end{Sinput}
\begin{Soutput}
[1] 4
\end{Soutput}
\begin{Sinput}
> A1 <- ((1/maxlambda) * A)%^%1
> A2 <- ((1/maxlambda) * A)%^%2
> A3 <- ((1/maxlambda) * A)%^%3
> A4 <- ((1/maxlambda) * A)%^%4
> A5 <- ((1/maxlambda) * A)%^%5
> A6 <- ((1/maxlambda) * A)%^%6
> A7 <- ((1/maxlambda) * A)%^%7
> A8 <- ((1/maxlambda) * A)%^%8
> A9 <- ((1/maxlambda) * A)%^%9
> A10 <- ((1/maxlambda) * A)%^%10
> A1
\end{Sinput}
\begin{Soutput}
      [,1] [,2]  [,3]
[1,]  0.25 0.75 -0.75
[2,] -0.75 1.75 -0.75
[3,] -1.50 1.50 -0.50
\end{Soutput}
\begin{Sinput}
> A2
\end{Sinput}
\begin{Soutput}
       [,1]  [,2]   [,3]
[1,]  0.625 0.375 -0.375
[2,] -0.375 1.375 -0.375
[3,] -0.750 0.750  0.250
\end{Soutput}
\begin{Sinput}
> A3
\end{Sinput}
\begin{Soutput}
        [,1]   [,2]    [,3]
[1,]  0.4375 0.5625 -0.5625
[2,] -0.5625 1.5625 -0.5625
[3,] -1.1250 1.1250 -0.1250
\end{Soutput}
\begin{Sinput}
> A4
\end{Sinput}
\begin{Soutput}
         [,1]    [,2]     [,3]
[1,]  0.53125 0.46875 -0.46875
[2,] -0.46875 1.46875 -0.46875
[3,] -0.93750 0.93750  0.06250
\end{Soutput}
\begin{Sinput}
> A5
\end{Sinput}
\begin{Soutput}
          [,1]     [,2]      [,3]
[1,]  0.484375 0.515625 -0.515625
[2,] -0.515625 1.515625 -0.515625
[3,] -1.031250 1.031250 -0.031250
\end{Soutput}
\begin{Sinput}
> A6
\end{Sinput}
\begin{Soutput}
           [,1]      [,2]       [,3]
[1,]  0.5078125 0.4921875 -0.4921875
[2,] -0.4921875 1.4921875 -0.4921875
[3,] -0.9843750 0.9843750  0.0156250
\end{Soutput}
\begin{Sinput}
> A7
\end{Sinput}
\begin{Soutput}
           [,1]      [,2]       [,3]
[1,]  0.4960938 0.5039063 -0.5039063
[2,] -0.5039063 1.5039063 -0.5039063
[3,] -1.0078125 1.0078125 -0.0078125
\end{Soutput}
\begin{Sinput}
> A8
\end{Sinput}
\begin{Soutput}
           [,1]      [,2]        [,3]
[1,]  0.5019531 0.4980469 -0.49804688
[2,] -0.4980469 1.4980469 -0.49804688
[3,] -0.9960938 0.9960938  0.00390625
\end{Soutput}
\begin{Sinput}
> A9
\end{Sinput}
\begin{Soutput}
           [,1]      [,2]         [,3]
[1,]  0.4990234 0.5009766 -0.500976563
[2,] -0.5009766 1.5009766 -0.500976563
[3,] -1.0019531 1.0019531 -0.001953125
\end{Soutput}
\begin{Sinput}
> A10
\end{Sinput}
\begin{Soutput}
           [,1]      [,2]          [,3]
[1,]  0.5004883 0.4995117 -0.4995117188
[2,] -0.4995117 1.4995117 -0.4995117188
[3,] -0.9990234 0.9990234  0.0009765625
\end{Soutput}
\end{Schunk}

Is $A$ a primitive matrix?  Take its fifth power to test it.

\begin{Schunk}
\begin{Sinput}
> B <- A%^%5
> B
\end{Sinput}
\begin{Soutput}
      [,1] [,2] [,3]
[1,]   496  528 -528
[2,]  -528 1552 -528
[3,] -1056 1056  -32
\end{Soutput}
\end{Schunk}




\end{document}
